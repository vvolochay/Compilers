\section{Грамматика}

\setlength{\grammarparsep}{20pt plus 1pt minus 1pt} % increase separation between rules
\setlength{\grammarindent}{12em} % increase separation between LHS/RHS 

\begin{grammar}

<prog> ::= <toplevel>*

<toplevel> ::= <id> <id> `;'
\alt <id> <id> `(' <args> `)' `;'
\alt <id> <id> `(' <args> `)' `{' <stmt>* `}'

<stmt> ::= `{' <stmt>* `}'
\alt <id> <id> `;'
\alt <id> `=' <expr> `;'
\alt <expr> `;'
\alt `if' `(' <expr> `)' <stmt> `else' <stmt>
\alt `while' `(' <expr> `)' <stmt>
\alt `return' <expr> `;'

<expr> ::= <id> 
\alt <num>
\alt <bool>
\alt `(' <expr> `)'
\alt <expr> `+' <expr>
\alt <expr> `-' <expr>
\alt <expr> `*' <expr>
\alt <expr> `<' <expr>
\alt <expr> `>' <expr>
\alt <expr> `==' <expr>
\alt <expr> `!=' <expr>
\alt <expr> `<=' <expr>
\alt <expr> `>=' <expr>
\alt <expr> `&&' <expr>
\alt <expr> `||' <expr>
\alt <id> `(' <ids> `)'

<args> ::= $\varepsilon$
\alt <id> <id>
\alt <id> <id> `,' <args>

<ids> ::= $\varepsilon$
\alt <id>
\alt <id> `,' <ids>

<bool> ::= `true' | `false'

<id> ::= <alpha>+

<num> ::= <digit>+

\end{grammar}

\section{Некоторые факты об этом компиляторе}

Компилируемый язык похож на очень сильно урезанный Си. Компиляция
однопроходная в том смысле, что нельзя использовать функции до их
определения. Компилируемая функция добавляется в контекст,
следовательно, поддерживаются рекурсивные функции. Можно делать
forward declaration, и, следовательно, косвенную рекурсию. 

Пока есть два типа: \texttt{int} и \texttt{bool}, компилятор проверяет
типы. Можно создавать локальные и глобальные переменные.
Глобальные переменные, также, как и функции, нельзя использовать до их
объявления. 

Конвенция вызова такая же, как в Си: первые четыре аргумента
передаются в r0--r3, остальные~--- в стеке.
Скомпилированные функции можно вызывать из Си. 
Можно вызывать функции из Си.

Скомпилированный код не содержит никаких оптимизаций и содержит
некоторые деоптимизации (для удобства копирует переданные в регистрах
аргументы в стек).

Код компилируется в стековую машину: вычисляются подвыражения, их
результаты снимаются со стека, производится вычисление и результат
кладётся обратно в стек.

\section{Примеры кода}

Эти примеры содержатся в папке \texttt{demo} в репозитории.

\subsection{Вычисление факториала}
 
\inputminted{c}{../demo/fact.l}

\subsection{Работа с глобальными переменными}

\inputminted{c}{../demo/global.l}

\subsection{Вычисление числа простых чисел на отрезке}

\inputminted{c}{../demo/isPrime.l}

\subsection{Двустороннее взаимодействие с Си}

\inputminted{c}{../demo/callc.l}

Код на Си:

\inputminted{c}{../demo/call.c}

\section{Как запускать}

На \texttt{akomarov.org} на 2222 порту есть машина на ARM-е, на
которой можно запускать примеры.

\begin{itemize}
\item Пользователь: \texttt{user}
\item Пароль: \texttt{ieti9Vai}
\end{itemize}

К сожалению, там установлена старая версия GHC, которой компилятор не
компилируется, поэтому, получается этакая кросс-компиляция.

\section{Что дальше}

\begin{itemize}
\item Добавить строки (и массивы).
\item Добавить возможность компилироваться в самостоятельный бинарник.
\item Переписать всё, потому что сейчас мерзко.
\end{itemize}
